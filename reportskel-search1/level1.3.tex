\subsection{Level 1.3: モデルの良さを評価する方法について}
\subsubsection{課題説明}
Level 1.2 で検討したモデルの適切さを評価する指標について検討した。

\subsubsection{評価に用いる情報源}
評価を用いる情報源としては,班員で話し合った後,確かめた.
\subsubsection{評価手順}
評価手順としては数値が大きいほど,信憑性が薄く,数値が小さいほど信憑性が高いものとなっている.
\subsubsection{評価に基づいた適切さを計る方法}
モデルに対して,それぞれの点から直線に対して直角に引いた差分を平均してその正確さを求めるという方法が出た.

例えば,その平均の値が大きいのならば,モデルに対して差分が大きいということなので,そのモデルの信憑性は薄れるということになり,逆に小さければそのモデルの信憑性が高いということになる.

