\subsection{Level 1.1: コンピュータと人間の違いを述べよ}
\subsubsection{課題説明}
コンピュータが人間より得意とするモノ、その反対に人間より不得手のモノ、両者について2つ以上の視点(立場や観点など)を示し、考察する。

\subsubsection{考察}
ロボットが得意なもの
\begin{itemize}
 \item データ処理など,既存のものをどうにかするという作業 \\
 \item 同時に複数の処理や,仕事をすることができる \\
 \item 離れていても,ネットワークが繋がっていたら遠くとのデータの処理をすばやくできる\\
 \item 演算処理が得意 \\
 \item 記憶力がある \\
\end{itemize}
ロボットが不得意なもの
\begin{itemize}
\item 全くの新しいものを一から作る\\
\item 電源が付いてなかったら何もできない \\
\item 運用コストが大きい \\
\item 移動できない \\
\item 欠損の可能性 \\

\item 感情に対する行動 \\
\item 命令しないと動かない \\
\item 途中結果を飛ばすということができない \\
\end{itemize}

やはりコンピュータは人間に比べて,演算能力ははるかに高く,膨大な量のデータの処理等は得意であるように感じる.それに関連して,同時に複数の作業を行うことができるというのも,コンピュータが得意とすることだろう.

それに比べて,コンピュータというものは人間に命令されなければ動くことができず,一から,新しいものを考えて作る,もしくは想像するというのは不得意のようだ.最近ではペッパーなど,多少移動ができるロボットやコンピュータがあらわれているが,やはりそもそもとして電源が付いてないと何も出来ないというところはロボットの不得意とするところであると考える.