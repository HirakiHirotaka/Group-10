\subsection{課題説明}
3種類の連続関数$y=x^2$、$z=x^2+y^2$、$y=-x \times \cos(x)$について、
最急降下法の適用を通して探索挙動を観察した。
以下ではまず共通部分である最急降下法の探索手続きについて、
フローチャートを用いて解説する。
その後、3種類の関数毎にプログラムの変更箇所、
観察意図観察方法、観察結果、考察について説明する。


\subsection{Level 2共通部分}
(補足:Level2.1, 2.2, 2.3 には共通する部分が多いため、
共通部分は独立して報告すると良いでしょう)

\subsubsection{探索の手続き(共通部分)}

\subsubsection{フローチャート(共通部分)}
(手続きとフローチャートはまとめて一つの節にしても構いません)

