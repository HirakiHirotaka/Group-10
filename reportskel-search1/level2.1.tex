\subsection{Level2.1: $y=x^2$ について}
\subsubsection{プログラムソース(変更部分)}

\subsubsection{観察意図と観察方法}
刻み値をより小さくすると探索点がより細かく移動するため最適性は良くなるだろう。しかし,小さくしすぎると探索回数が増えてしまい,効率性は悪くなるだろう。逆に刻み値を大きくすると探索回数は減って効率性はよくなるだろう。しかし,その分最適性は悪くなるだろう。

上記のように予想してこれを検証するために,seed値を1に固定しalphaの値を変えることにより探索を行い,結果を観測する。
\subsubsection{実行結果}
\subsubsection{考察}

