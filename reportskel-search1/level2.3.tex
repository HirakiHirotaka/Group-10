\subsection{Level2.3: $y=x*\cos(x)$ について}
Level2.3では $y=x*\cos(x)$ を最急降下法で探索する.
最急降下法では微分した値をアルゴリズムで必要とする為まずはこのモデル式の微分を導出する.

\begin{align}
    y &= x * \cos(x) \label{alig:cos}\\
    y' &= \cos(x) - x * \sin(x)
    \label{alig:xcosd}
\end{align}


\subsubsection{プログラムソース(変更部分)}

上記 (\ref{alig:xcosd})式で導出した微分値を利用してsteepest-decent.cを以下の様に変更した. (コード:\ref{code:level23})
尚変更点の列挙には元のソースコードが入っていたsteepestsearch2-2/steepest\_decent.cと今回のコードをdiffコマンドを用いた.
変更した箇所を+記号で示している.

\lstinputlisting[caption=変更点,label=code:level23]{../steepestsearch2-3/diff.txt}
軽微な修正であるが,y軸に対して設定するは値域であると考えたので,定義域から値域に変更した.
続いて,後述するが引数を1つ以上取るように設定したので,usageのメッセージを変更した.

diff結果の29行目以降では,モデル式の表現として,C言語のcos関数とxを乗算し,結果をzに代入する様にコードを変更した.
36行目から示す関数pd\_x及びpd\_yの変更点であるが,まず $y=x* \cos(x)$を実験班で偏微分したところ,得られた解を表現した.
実際にxで偏微分をした結果を代入するz\_dxには,cos,sin関数をそれぞれ利用している.pd\_yは偏微分した結果が0であるので
初期状態のまま手を付けなかった.


学習係数alphaを引数として変更できる用,argvのエラーメッセージを選択するif文を変更した.
コマンドライン引数の総数が2つある場合,第2引数としてalphaを受取り,char型からdouble型への変換を行っている.


\subsubsection{実験意図と観察方法}
最初に学習係数が探索に影響をあたえる物が何かを決定づけるため,seedを固定して学習係数Alphaのみ変更を行う.
実験では,最初に効率に着目しstepの回数を比較する.続いて,正確性を判断するためにシェルスクリプトを用いてモデル式の上に結果をplotし,確認を行った.

続いてseedがどのような影響を与えるかを観察する為,Alpha値を固定しseedを変更して観察を行った.内容はAlphaと同様に,step回数,及びモデル式との比較を行った.

\subsubsection{Alphaの変更実験}

まずAlphaの値を変更した場合,step数にどう変化があるかを実験する.
今回はシェルスクリプトを用いて自動化し,実験を行った.

スクリプトはrun\_ave.shを一部改変したrun\_ave\_alpha.shを利用した為,本レポートでは変更点を示す.
尚ファイル名はaverageが入っているが,今回は平均値を取ってはおらず,複数回のループ処理の雛形として流用した.

\lstinputlisting[caption=run\_ave.shとrun\_ave\_alpha.shの差分,label=code:runave]{../steepestsearch2-3/shell_dif.log}

今回は結果を出力するテキストファイルを変数result\_file,gnuplotでの出力pdf名をpdf\_titleでそれぞれ設定している.
また,seed値を固定し,マジックナンバーを使用しない為に,変数seedに1000を設定している.

元のシェルスクリプトを参考にし,結果を書き出すテキストファイルがあれば削除するように変更した.
シェルスクリプト内でループをさせる場合,全要素を書き込んでループさせる事も考えられる.
しかし,今回は10区切りで20パターン計測をする為に全要素を書き込むのは冗長と判断した.
その為,while分で初期化しながら,配列roops\cite{shellq}を利用した.

今回は配列roopsに収納した変数20個分.つまり20回ループし
実際にグラフ化したものを図 に示す.
その為,外れ値と思われるものを手作業で削除し,再plotしたものを図 に示す.

\subsubsection{seedの変更実験}

\subsubsection{実行結果}
\subsubsection{考察}

