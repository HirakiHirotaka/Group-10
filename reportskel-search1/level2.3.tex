\subsection{Level2.3: $y=x*\cos(x)$ について}
Level2.3では $y=x*\cos(x)$ を最急降下法で探索する.
最急降下法では微分した値をアルゴリズムで必要とする為まずはこのモデル式の微分を導出する.

\begin{align}
    y &= x * \cos(x) \label{alig:cos}\\
    y' &= \cos(x) - x * \sin(x)
    \label{alig:xcosd}
\end{align}


\subsubsection{プログラムソース(変更部分)}

上記 (\ref{alig:xcosd})式で導出した微分値を利用してsteepest-decent.cを以下の様に変更した. (コード:\ref{code:level23})
尚変更点の列挙には元のソースコードが入っていたsteepestsearch2-2/steepest\_decent.cと今回のコードをdiffコマンドを用いた.
変更した箇所を+記号で示している.

\lstinputlisting[caption=変更点,label=code:level23]{../steepestsearch2-3/diff.txt}
軽微な修正であるが,y軸に対して設定するは値域であると考えたので,定義域から値域に変更した.
続いて,後述するが引数を1つ以上取るように設定したので,usageのメッセージを変更した.

diff結果の29行目以降では,モデル式の表現として,C言語のcos関数とxを乗算し,結果をzに代入する様にコードを変更した.
36行目から示す関数pd\_x及びpd\_yの変更点であるが,まず $y=x* \cos(x)$を実験班で偏微分したところ,得られた解を表現した.
実際にxで偏微分をした結果を代入するz\_dxには,cos,sin関数をそれぞれ利用している.pd\_yは偏微分した結果が0であるので
初期状態のまま手を付けなかった.


学習係数alphaを引数として変更できる用,argvのエラーメッセージを選択するif文を変更した.
コマンドライン引数の総数が2つある場合,第2引数としてalphaを受取り,char型からdouble型への変換を行っている.


\subsubsection{実験意図と観察方法}
最初に学習係数が探索に影響をあたえる物が何かを決定づけるため,seedを固定して学習係数Alphaのみ変更してまず結果を確認した.
続いてseedがどのような影響を与えるかを観察する為,Alpha値を固定しseedを変更して観察を行った.

\subsection{Alphaの変更実験}

\subsubsection{実行結果}
\subsubsection{考察}

