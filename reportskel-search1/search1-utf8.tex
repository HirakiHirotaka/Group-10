%\documentclass[a4paper,10pt]{jarticle}
%\documentclass[]{jarticle}
\documentclass[uplatex,10pt]{jsarticle}

%\usepackage{graphicx}
\usepackage[dvipdfmx]{graphicx}
\usepackage{eclbkbox} %breakbox用
\usepackage[T1]{fontenc}
\usepackage{lmodern}
\usepackage{amsmath}
\usepackage{ascmac} %shedebox用
\usepackage{here} %図を好きな位置に
\usepackage{here}
\usepackage{listings, jlisting}
\lstset{%
  language={c},%言語名前
  basicstyle={\footnotesize\ttfamily},%ソースコードのフォント設定
  identifierstyle={\small},
  ndkeywordstyle={\small},
  keywordstyle={\ttfamily},
  stringstyle={\small\ttfamily},
  frame={tb},
  breaklines=true,%行が長くなった際の改行の有無
  xrightmargin=0zw,%右の余白の大きさ
  xleftmargin=3zw,%左の余白の大きさ
  numbers=left,%行番号表示場所
  stepnumber=1,%行番号の増分
  numbersep=1zw,%行番号と本文の間隔
  numberstyle=\ttfamily,
  frame=tRBl,
  framesep=5pt,
  commentstyle={\ttfamily},%コメントアウトのフォント設定
  flexiblecolumns = true,
  classoffset=1,
  showstringspaces=false,
  tabsize=4
}

%本文領域を広め(空白箇所マージン領域を小さめ)に設定
\setlength{\textwidth}{179mm}
\setlength{\textheight}{251mm}
\setlength{\topmargin}{-2cm}
\setlength{\oddsidemargin}{-1cm}
\setlength{\evensidemargin}{-1cm}

\begin{document}

\title{情報工学実験IIレポート(探索アルゴリズム1)}
\author{曜日&グループ番号: 金曜日&グループ9} %
\date{実施日:2017年1月20日 (金)、提出日:2017年2月0日}

\maketitle


\section*{グループメンバ}
今回の実験はグループメンバ全員で取り組んだが,レポート化にあたって分担を行った.
グループメンバと担当した分野は次の通りである.
\begin{itemize}
 \item 155706J 久場翔悟: 担当Level2.1
 \item 155711E 平木宏空: 担当Level1.1,1.2,1.3
 \item 155716F 石塚海斗: 担当Level2.2
 \item 155730B 清水隆博: 担当Level2,Level2.3
\end{itemize}

\section*{提出したレポート一式について}
レポート一式は
``\verb|shell:/net/home/teacher/tnal/2016-search1-mon/group0/|''
にアップロードした。
提出したファイルのディレクトリ構成は以下の通りである。

\vspace{+0.5cm}
\begin{breakbox}
\begin{verbatim}
./src/      # 作成したプログラム一式
./report/   # レポート関係ファイル.図ファイルを含む.
./steepestsearch2-1/ #Level2.1で作成したプログラム一式
\end{verbatim}
\end{breakbox}

\newpage

\section{Leve l1: 最適化とは}
\subsection{Level 1.1: コンピュータと人間の違いを述べよ}
\subsubsection{課題説明}
コンピュータが人間より得意とするモノ、その反対に人間より不得手のモノ、両者について2つ以上の視点(立場や観点など)を示し、考察する。

\subsubsection{考察}
ロボットが得意なもの
\begin{itemize}
 \item データ処理など,既存のものをどうにかするという作業 \\
 \item 同時に複数の処理や,仕事をすることができる \\
 \item 離れていても,ネットワークが繋がっていたら遠くとのデータの処理をすばやくできる\\
 \item 演算処理が得意 \\
 \item 記憶力がある \\
\end{itemize}
ロボットが不得意なもの
\begin{itemize}
\item 全くの新しいものを一から作る\\
\item 電源が付いてなかったら何もできない \\
\item 運用コストが大きい \\
\item 移動できない \\
\item 欠損の可能性 \\

\item 感情に対する行動 \\
\item 命令しないと動かない \\
\item 途中結果を飛ばすということができない \\
\end{itemize}

やはりコンピュータは人間に比べて,演算能力ははるかに高く,膨大な量のデータの処理等は得意であるように感じる.それに関連して,同時に複数の作業を行うことができるというのも,コンピュータが得意とすることだろう.

それに比べて,コンピュータというものは人間に命令されなければ動くことができず,一から,新しいものを考えて作る,もしくは想像するというのは不得意のようだ.最近ではペッパーなど,多少移動ができるロボットやコンピュータがあらわれているが,やはりそもそもとして電源が付いてないと何も出来ないというところはロボットの不得意とするところであると考える.
\subsection{Level 1.2: 住宅価格を推定するモデルについて}
\subsubsection{課題説明}
Housing Data Set\cite{housingdata}
におけるRM(平均部屋数)からMEDV(平均価格)を推定するた
めのモデルについて検討した。

\subsubsection{モデル}
自分たちの班では一次関数をグラフの中に引くという案が出た.点が密集している部分を通るように,一次関数を引くという方法である.


\subsubsection{モデルへの入力}
平均価格の数値をこの一次関数に入力.
\subsubsection{モデルにおける処理内容}
一次関数は入力された数値に比例する値を出力する.
\subsubsection{モデルの出力}
出力された値というのは,平均価格に対する部屋数を表している.
\subsubsection{問題点}
この方法はある程度の比較された値を求めることが可能であるが,その一次関数から著しく離れた値を切り捨てることになるのでそのあたりの数値に関して信憑性が保てないという問題点がある.

% (補足:PDF図を挿入する例)

% \begin{figure}[h]
%  \begin{center}
%   \includegraphics[width=8.0cm]{figs/system-image.pdf}
%   \caption{入出力と内部モデルのイメージ図}
%  \end{center}
% \end{figure}


\subsection{Level 1.3: モデルの良さを評価する方法について}
\subsubsection{課題説明}
Level 1.2 で検討したモデルの適切さを評価する指標について検討した。

\subsubsection{評価に用いる情報源}
評価を用いる情報源としては,班員で話し合った後,確かめた.
\subsubsection{評価手順}
評価手順としては数値が大きいほど,信憑性が薄く,数値が小さいほど信憑性が高いものとなっている.
\subsubsection{評価に基づいた適切さを計る方法}
モデルに対して,それぞれの点から直線に対して直角に引いた差分を平均してその正確さを求めるという方法が出た.

例えば,その平均の値が大きいのならば,モデルに対して差分が大きいということなので,そのモデルの信憑性は薄れるということになり,逆に小さければそのモデルの信憑性が高いということになる.



\newpage

\section{Level 2: 最急降下法による最適化}
\subsection{課題説明}
3種類の連続関数$y=x^2$、$z=x^2+y^2$、$y=-x \times \cos(x)$について、
最急降下法の適用を通して探索挙動を観察した。
以下では,最急降下法のアルゴリズムについて
フローチャートを用いて解説する。
その後、3種類の関数毎にプログラムの変更箇所、
観察意図観察方法、観察結果、考察について説明する。


\subsection{Level 2共通部分}

\subsubsection{最急降下法のアルゴリズム}
最急降下法とは
\begin{oframed}
微分値を基にxを移動させる幅を決定する際
\begin{align*}
x_{next} = x - \alpha * f'(x)
\end{align*}
によって移動先を算出するアルゴリズムである.\cite{info2-search1}
\end{oframed}

微分値はその対象のモデル式における切片の傾きに等しい.
つまり,ある点から切片を引き,そこに対して学習係数文移動させる方法である.
最急降下法の名前の通り,関数の最小値へと引き寄せられるように移動していく.
尚最大値を求める場合は,学習係数$\alpha$を符号反転させれば良い.
この際,探索点が移動した場所を可視化すると,より良い精度の場合はモデル式と同じグラフを辿ると考える.

また最急降下法を用いた今回のCプログラムsteepest\_decent.cについてフローチャート図を用いて解説を行う. (図:\ref{fig:flowsttep})

\begin{figure}[H]
\centering
\includegraphics[width=0.5\textwidth]{figs/flowchart.pdf}
\caption{steepest\_decent.cのフローチャート図}
\label{fig:flowsttep}
\end{figure}

このCプログラムは,フローチャート図に示す通り探索場所を移動する度に定義域,mた前回との探索場所の移動について分岐させ,最適解を見つけ出すプログラムである.
最急降下法を用いるにあって,このプログラムではプログラマがソースコード中にモデル式及び偏微分を行った結果を入力する必要がある.
今回の実験ではlevel2の課題はこのCプログラムをそのまま,又は一部変更して利用した.

 %共通部分の結果及び考察
\subsection{Level2.1: $y=x^2$ について}
\subsubsection{プログラムソース(変更部分)}

\subsubsection{観察意図と観察方法}
刻み値をより小さくすると探索点がより細かく移動するため最適性は良くなるだろう。しかし,小さくしすぎると探索回数が増えてしまい,効率性は悪くなるだろう。逆に刻み値を大きくすると探索回数は減って効率性はよくなるだろう。しかし,その分最適性は悪くなるだろう。

上記のように予想してこれを検証するために,seed値を1に固定しalphaの値を変えることにより探索を行い,結果を観測する。
\subsubsection{実行結果}
\subsubsection{考察}


\subsection{Level2.2: $z=x^2 + y^2$ について}
\subsubsection{フローチャート共通からの変更点}
今回,Level2.2 に取り組むにあたって,プログラムの変更を行ったため,
フローチャートにも多少の変更が出た。\\
変更点を以下の図\ref{flow2.2}に示す。\\
  \begin{figure}[H]
	\begin{center} %センタリングする
	  \includegraphics[scale=0.45]{./flowchart2-2.pdf}
	  \caption{フローチャート変更点} %タイトルをつける
	  \label{flow2.2} %ラベルをつけ図の参照を可能にする
	\end{center}
  \end{figure}


\subsubsection{プログラムソース(変更部分)}
以下の図\ref{cp22}に変更部分のみを示す。
	\begin{figure}[H]
        \caption{Level2.2 変更点}
		\label{cp22}
		\fontsize{10pt}{10pt}\selectfont
      	\begin{shadebox}
        	\begin{verbatim}
			//main 関数直後
			f( argc != 3 ){
			(略)
			}else{
			(略)
			alpha = atof(argv[2]); 
			(略)
			}
			
			//f 関数内
			//  z = x;
			  z = x*x + y*y;

			//pd_x 関数内
			//  z_dx = 1;
			  z_dx = 2*x;

			//pd_y 関数内
			//  z_dy = 0;
			  z_dy = 2*y;
        	\end{verbatim}
      	\end{shadebox}
     \end{figure}

\subsubsection{観察意図と観察方法}
seed値を固定してalpha 値を変動させることで,探索点の刻み幅による探索の最適性及び
効率性を検討する。
alpha 値を変更することで探索幅も変動することから,以下の2つのような予想が立てられる。
1, alpha 値が大きければ探索点の刻み幅も大きくなり,効率性を向上できるが,最適性が低下
する。
2, alpha 値が小さければ探索点の刻み幅が小さくなるため,最適性を向上できるが,効率性
が低下する。\\
効率性の観察方法は,seed値を固定しalpha 値を変動させた際のstep 数の推移を
各seed 値(範囲1000-10000 1000刻み)ごとに表して行う。
効率性の検討はグラフより,各seed 数で最小step 数のalpha 値を読み取り,
最も優れたalpha 値を多数決で決定し,それをこのプログラムの最大効率であると決定,
改善点を考察する。\\
また,srand 関数にコマンドライン引数のseed 値が渡されているため,
rand 値は同じseed 値 を入力している限り,一定である。よって,同一seed 値 において,
乱数を考慮しての複数実行,実行結果の平均値取得等はしない。\\
最適性の観察方法は,seed値を固定しalpha 値を変動させた際の終了時座標,これと
傾きが0になっている座標(ここでは0,0)との差の推移を各seed 値
(範囲1000-10000 1000刻み)ごとにグラフで表す。
最適性の検討はグラフからその差が最小のalpha の値を各seedから読み,最も優れた
alpha 値を多数決で決定する。
そのalpha 値をこのプログラムの最大最適値 であると決定し,改善点を考察する。

\subsubsection{実行結果}
  \begin{enumerate}
  \renewcommand{\labelenumi}{\arabic{enumi}}

  \item 効率性について\\
	効率性の観察のための手法として,各seed値(1000-10000の1000刻み10種),
	各alpha値(0.001 と 0.1-1.0(0.1刻み)の計11種) ごとに
	最終step数をplot していき,その推移傾向を観察することで行った。\\
	以下の図\ref{stepgraph1}がその結果である。

  \begin{figure}[H]
	\begin{center} %センタリングする
	  \includegraphics[width=9cm]{../steepestsearch2-2/createStepGraph/StepGraph.png}
	  \caption{各alpha値(0.001-1.0) での終了step数} %タイトルをつける
	  \label{stepgraph1} %ラベルをつけ図の参照を可能にする
	\end{center}
  \end{figure}

  alpha 値が 0.001, 1.0 の時に step数が1000となり,alpha 値 0.1-0.9 までの
  点の差が見えづらい。よって,alpha 値が0.1-0.9 の範囲の最終step数のグラフも
  図\ref{stepgraph2}に用意した。
  \begin{figure}[H]
	\begin{center} %センタリングする
	  \includegraphics[width=9cm]{../steepestsearch2-2/createStepGraph/StepGraph2.png}
	  \caption{各alpha値(0.1-0.9) での終了step数} %タイトルをつける
	  \label{stepgraph2} %ラベルをつけ図の参照を可能にする
	\end{center}
  \end{figure}


  \item 最適性について\\
	最適性の観察のための手法として,各seed値(1000-10000の1000刻み10種),
	各alpha値(0.001 と 0.1-1.0(0.1刻み)の計11種) ごとに
	終了座標の誤差(x,yの合計)をplot していき,その推移傾向を観察することで行った。\\
	以下の図\ref{diffgraph1}がその結果である。


  \begin{figure}[H]
	\begin{center} %センタリングする
	  \includegraphics[width=9cm]{../steepestsearch2-2/createDiffGraph/DiffGraph.png}
	  \caption{各alpha値(0.001-1.0) での終了座標誤差(x,y合計)} %タイトルをつける
	  \label{diffgraph1} %ラベルをつけ図の参照を可能にする
	\end{center}
  \end{figure}

  alpha 値が 0.001, 1.0 の時に 最大合計誤差10を超え,alpha 値 0.1-0.9 までの
  点の差が見えづらい。よって,alpha 値が0.1-0.9 の範囲の合計誤差のグラフも
  図\ref{stepgraph2}に用意した。
  \begin{figure}[H]
	\begin{center} %センタリングする
	  \includegraphics[width=9cm]{../steepestsearch2-2/createDiffGraph/DiffGraph2.png}
	  \caption{各alpha値(0.1-0.9) での終了座標誤差(x,y合計)} %タイトルをつける
	  \label{diffgraph2} %ラベルをつけ図の参照を可能にする
	\end{center}
  \end{figure}

  \end{enumerate}

\subsubsection{考察}
まず,効率性についての考察を行う。\\
図\ref{stepgraph1}より,最終step数が1000回以内に収まるのが,$0.001 < alpha < 1.0$
の範囲であることがわかる。このことから,alpha 値が小さすぎると探索できる
範囲が狭まり,最適解まで届かなかったと推察する。 図\ref{stepgraph2}からは,
総step数が1番低い点のalpha 値が 0.5 であり,0.5 から離れるごとにstep 数が
増加する傾向にあることがわかる。alpha値が0.5 の時に総step数が最小となった理由を,
プログラム内の探索点移動に用いられている式より考察する。\\
$x = x - alpha*pd_x(x,y);$\\
この式のpd\_x(x,y) はxについての偏微分であるため,置き換えると\\
$x = x - alpha*2x;$\\
となる。この式に alpha = 0.5 を代入すると x には '0' が入ることとなるため,
移動後のx座標がちょうど傾きが0になる点(最終的に移動したい最適解)になる。
yについても同様にしてalpha = 0.5 の時に1回移動後の座標は'0'となる。\\
このことから, alpha = 0.5 という値は 今回探索した $x^2 + y^2$ の式において
あらゆる座標から最適解の座標を求めることができる値であることがわかる。\\

次に,最適性について\\
効率性の考察にて
図\ref{stepgraph1}より,最終step数が1000回となるのが$alpha=0.001 , alpha=1.0$
で,言い換えると$alpha=0.001 , alpha=1.0$のとき最適解との誤差が大きい。\\
このことが,終了座標誤差を表す図\ref{diffgraph1}からも読み取れる。
図\ref{diffgraph2}からは効率性と同じく alpha=0.5 に最適性が最も高いことがわかる。
さらにalpha=0.5 の点へのグラフの収束の度合いにも alpha が0.5より小さい時と
大きい時で差がある。これは,alphaが0.5より小さい場合の最適解からの誤差は探索点が
単純に最適解に届かなかったものであるが,alpha が0.5より大きい場合では
最適解付近まで探索点が到達したが探索点の移動幅の大きさが影響し最適解には
至らなかったというもの,という差によるものと推察する。


\subsection{Level2.3: $y=x*\cos(x)$ について}
Level2.3では $y=x*\cos(x)$ を最急降下法で探索する.
最急降下法では微分した値をアルゴリズムで必要とする為まずはこのモデル式の微分を導出する.

\begin{align}
    y &= x * \cos(x) \label{alig:cos}\\
    y' &= \cos(x) - x * \sin(x)
    \label{alig:xcosd}
\end{align}


\subsubsection{プログラムソース(変更部分)}

上記 (\ref{alig:xcosd})式で導出した微分値を利用してsteepest-decent.cを以下の様に変更した.

まず学習係数alphaを引数として変更できる用,argvのエラーメッセージを選択するif文を変更した.
第二引数としてalphaを受取り,char型からdouble型への変換を行っている.


\subsubsection{観察意図と観察方法}
今回は学習係数が探索に影響をあたえる物が何かを決定づけるため,seedではなく初期値位置を固定した状態で探索を行った.
探索にはシェルスクリプトを用いた.

\subsubsection{実行結果}
\subsubsection{考察}



\vspace{+1.0cm}
(補足:参考文献は thebibliography 環境を使って列挙し、
本文中で適切な箇所で引用するようにしましょう。
例えば下記文献は、アブストラクトやLevel 4で引用しています)
\begin{thebibliography}{99}
\bibitem{info2-search1}
情報工学実験2: 探索アルゴリズムその1(當間)\\
\verb|http://www.eva.ie.u-ryukyu.ac.jp/~tnal/2016/info2/search1/|
\bibitem{housingdata}
Housing Data Set\\
\verb|http://archive.ics.uci.edu/ml/datasets/Housing|
\end{thebibliography}

\end{document}
